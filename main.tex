%%%%%%%%%%%%%%%%%%%%%%%%%%%%%%%%%%%%%%%%%%%%%%%%%%%%%%%%%%%
%% Congratulations, you've made an excellent choice
%% of writing your Tampere University thesis using
%% the LaTeX system. This document attempts to be
%% as complete a template as possible to let you focus
%% on the most important part: the writing itself.
%% Thus the details regarding the visual appearance
%% and even structure have already been worked out
%% for you!
%%
%% I sincerely hope you will find this template useful
%% in completing your thesis project. I've tried to
%% add comments (followed by the % sign) to clarify
%% the structure and purpose of some of the commands.
%% Most of the magic happens in the file tauthesis.cls,
%% which you are more than welcome to take a look at.
%% Just refrain from editing it in the most crucial
%% versions of the thesis!
%%
%% I wish you and your thesis project the best of luck!
%% If this template causes you trouble along the way
%% or if you've any suggestions for improving it,
%% please contact me via email at
%%
%% ville.koljonen (at) tuni.fi.
%%
%% Yours,
%%
%% Ville Koljonen
%% 16th May 2019
%%
%% PS. This template or its associated class file don't
%% come with a warranty. The content is provided as is,
%% without even the implied promise of fitness to the
%% mentioned purpose. You, as the author of the thesis,
%% are responsible for the entire work, including the
%% provided material. No one else is liable to you for
%% any damage inflicted on you or your thesis, were it
%% caused by using this template or not.
%%%%%%%%%%%%%%%%%%%%%%%%%%%%%%%%%%%%%%%%%%%%%%%%%%%%%%%%%%%


%%%%% INSTRUCTIONS FOR COMPILING THE DOCUMENT %%%%%
%% Overleaf: just click Recompile.
%% Terminal:
%%  1. pdflatex main.tex
%%  2. makeindex -s main.ist -t main.glg -o main.gls main.glo
%%  3. biber main
%%  4. pdflatex main.tex
%%  5. pdflatex main.tex
%% Similar sequence of commands is also required
%% in LaTeX specific editors.
%%%%%%%%%%%%%%%%%%%%%%%%%%%%%%%%%%%%%%%%%%%%%%%%%%%

%%%%% PREAMBLE %%%%%

\nonstopmode
\documentclass{tauthesis}

% The glossaries package throws a warning:
% No language module detected for 'finnish'.
% You can safely ignore this. All other
% warnings should be taken care of!

%%%%% Your packages.
% Before adding packages, see if they can be found
% in tauthesis.cls already. If you're not sure that
% you need a certain package, don't include it in
% the document! This can dramatically reduce
% compilation time.

% Graphs
\usepackage{pgfplots}
\pgfplotsset{compat=1.15}

% Subfigures and wrapping text
\usepackage{subcaption}

% Mathematics packages

\usepackage{amsmath, amssymb, amsthm}
%\usepackage{bm}

%%%%% Your commands.

%%%%% Glossary information.

\loadglsentries[main]{tex/sanasto.tex}
\makeglossaries

%%%%% Citation information.

\addbibresource{tex/references.bib}

\hypersetup{hidelinks}

\begin{document}

%%%%% FRONT MATTER %%%%%

\frontmatter

%%%%% Thesis information and title page.

\title{Lämpöpumppujen kommunikaatiorajapinnat ja niiden hyödyntämismahdollisuudet}
\subtitle{Luonnos: versio 1}

% The author name.
\author{Ilmari Marttila, ilmari.marttila@tuni.fi}

% The examiner information.
% If your work has multiple examiners, replace with
% \examiner[<label>]{<name> \\ <name>}
% where <label> is an appropriate (plural) label,
% e.g. Examiners or Tarkastajat, and <name>s are
% replaced by the examiner names, each on their
% separate line.
\examiner{Prof. Sami Repo}

% The finishing date of the thesis (YYYY-MM-DD).
\finishdate{2020}{01}{29}


\thesistype{Kandidaatintyö}

\facultyname{Informaatioteknologian ja viestinnän tiedekunta}
\programmename{Tieto- ja sähkötekniikan TkK-tutkinto-ohjelma - Sähkötekniikka}

\keywords%
    {avainsana}

\maketitle

%%%%% Abstracts and preface.

%\abstract{tex/tiivistelma.tex}

%\preface{tex/alkusanat.tex}{Tampereella}

%%%%% Table of contents.

\tableofcontents

%%%%% Lists of figures, tables, listings and terms.

%\listoffigures
%\listoftables
%\lstlistoflistings


%\glossary

%%%%% MAIN MATTER %%%%%

\mainmatter
\chapter{Suunnitelma}
\section{Aiheen valinnan yhteydessä esitetty sisältöhahmotelma}
  Työssä voisin käsitellä esimerkiksi seuraavia aiheita.
  \begin{itemize}
    \item  Mikäli suuri määrä lämpöpumppuja olisi ohjattavissa keskitetysti, voidaan niitä käyttää virtuaalivoimalana tehonkulutushuippujen tasaamiseen. Monipuoliset rajapinnat voisivat mahdollistaa älykkään tehontarpeen priorisoinnin.
    \item Rajapintojen turvallisuus. Suojaamattomia yhteyksiä voidaan käyttää hyvän lisäksi myös pahaan.
    \item Listata ja vertailla olemassa olevien lämpöpumppujen sovelluskehittäjille tarjoamia rajapintoja.
    \item  Toimivatko lämmitysjärjestelmät nykyään luetettavasti myös ilman internet-yhteyttä? Entä tulevaisuudessa?
    \item Voisin esimerkiksi toteuttaa demotarkoitutksissa pienen käyttöliittymän johonkin valmiiseen lämpöpumppuun, mikäli sopiva ehdokas löytyy
  \end{itemize}
\section{Aiheanalyysi}

  Läpöpumppujen määrä kotitalouksissa on noussut runsaasti viimeisten vuosien aikana. Lämpöpumput nähdään ekologisena vaihtoehtona lämpöenergian tuottamiseen. Samaan aikaan sähköntuotanto on tietynlaisessa murroksessa, jossa keskitetystä tuotannosta siirrytään kohti hajautettua ja täten hankalammin ennustettavaa (ja säästä riippuvaa) sähöntuotantoa. Kun tuotannon vaihtelu lisääntyy, myös kulutuksen pitää olla joustavampaa.

  Selvitän työssäni mahdollisuuksia ja tapoja, joilla lämpöpumput voidaan ottaa mukaan tehontarpeen tasaukseen ja älykkään verkon toimintaan. Sitä ennen kuitenkin käydään läpi pumppujen sähkötekniikkaa ja sitä, miten pumput vaikuttavat sähköverkkoon. Teen myös katsauksen tapoihin ja teknologioihin, joilla pumppuja voidaan ohjata etänä. Olisi hienoa, jos saisin tehtyä jonkinlaisen pienen käytännön toteutuksen aiheesta, mutta en ole vielä löytänyt sopiavaa 

\section{sisältösuunnitelma}

\begin{enumerate}
  \item Johdanto
  \item Lämpöpumput sähkön käyttäjinä
    \begin{enumerate}
      \item Oikosulkumoottorit
      \item Taajuusmuuttajakäytöt
      \item Resistiivinen kuorma -- vastukset
    \end{enumerate}
  \item Rajapinnat -- eri abstraktiotasot
  \begin{enumerate}
    \item Smart Grid
    \item matalan tason kommunikointi
  \end{enumerate}
  \item lämpöpumppujen ohjauksella saavutettavat hyödyt
  \item käytännön toteutus tai simulointi
  \item yhteenveto
\end{enumerate}



\section{Aineistoja ja lähteitä}
  \begin{itemize}
    \item Perustietoa lämpöpumpuista. Luentokalvot kurssilta DEE-53100. \parencite{kummu}
    \item Yleiskästyksen luontia niistä ongelmista, mihin rajapintojen hyödyntämisellä voidaan vaikuttaa. \parencite[Luku 3]{rautiainen}
    \item Mielenkiintoista tietoa aiheesta. Johdattaa tutkimaan, mitä tarkoitetaan termillä \emph{SG-ready}. \parencite{ModelBasedFlexibilityAssessment}
    \item Käsitteleen myös muita kuorman tasaukseen liittyviä asioita \parencite{ShenJiangLi}
    \item Lisää aiheesta SG-ready \parencite{fischerTriebelSelinger}
    \item Modbus \parencite{sousaPortugal}
  \end{itemize}


%\chapter{Johdanto}
%\label{ch:johdanto}
%Lämpöpumppujen käyttö kiinteistöjen lämmitykseen kasvattaa suosiotaan. Vuoden 2000 jälkeen on Suomeen asennettu noin 700 000 lämpöpumppua\footnote{Alle \SI{26}{\kilo\watt}}\parencite{kummu}. Lämpöpumppujen kuluttama teho saattaa pienissä muuntopiireissä kasvaa hyvinkin merkittäväksi sen korvatessa vanhoja lämmitysmenetelmiä. Tämä aiheuttaa ongelmia sähkön siirrossa. Lämpöpumppuja, kuten monia muitakin sähköä käyttäviä laitteita, voidaan kuitenkin ohjata. On mahdollista toteuttaa paikallista kysyntäjoustoa ja ohjata lämpöpumppuja pientuotannon tuotantomäärien mukaan.

Lämpöpumppujen kasvanut suosio avaa myös uusia mahdollisuuksia. Suuria määriä lämpöpumppuja voidaan ohjata yhtenä ryhmänä. Ryhmää ohjaamalla voidaan toteuttaa kysyntäjoustoa jopa kansallisella tasolla ja vakauttaa säkövekon taajuutta ja jännitettä. Lämöpumppuryhmiä voidaan käyttää myös virtuaalivoimaloina ja niiden tehokapasiteettia voidaan myydä siitä maksaville\parencite{ShenJiangLi, fischerTriebelSelinger}.

Lämpöpumppuja on mahdollista siis käyttää osana älykästä sähköverkkoa. Lämpöpummpujen ohjaus voidaan mahdollistaa erilaisten rajapintojen kautta. Laitevalmistajien on tarjottava nämä rajapinnat. Rajapintoja voidaan toteuttaa monilla tavoilla ja valmiita teknisiä ratkaisuja on olemassa.


% Add chapters

%\chapter{Yhteenveto}
%\label{ch:yhteenveto}
%Lämpöpumppujen ja aurinkosähköjärjestelmien merkitys sähköverkossa kasvaa niiden suosion kasvaessa. Lämpöpumput korvaavat muita lämmitys- ja viilennysmuotoja ja aurinkosähköjärjestelmillä haetaan kotitalouksissa omavaraisuutta ja ylipäätään säästöjä sähkölaskussa.

Aurinkosähköjärjestelmien määrän kasvu lisää sähköverkon tuotannon ailahtelevuutta ja säätövoiman tarvetta, sillä tuotanto vaihtelee tunneittain. Tuotantoa voidaan ennustaa sääennusteen pohjalta, mutta yksittäiset pilvet voivat aiheuttaa alueellisesti nopeitakin tehovaihteluita. Lisäksi huipputeho sijoittuu päiväsaikaan, joten ihmisten palatessa töistä ja kulutuksen kasvaessa tuotannon laskiessa samanaikaisesti säätötehon tarve kasvaa entisestään.

Lämpöpumppujen lisääntyminen ja esimerkiksi suorasta sähkölämmityksestä eroava käyttäytyminen vaikuttavat sähköverkkoon. Lämpöpumppujen tekniikka kehittyy ja nykyisin kotitalouskäyttöön myytävissä pumpuissa on yleensä suoraan verkkoon kytketyn oikosulkumoottorin tilalla taajuusmuuttajakäyttöinen oikosulkumoottori. Tämä kehitys myös osaltaan määrittelee miten verkkoon liitettyjen lämpöpumppujen määrän kasvu vaikuttaa sähköverkkoon. Ilman taajuusmuuttajaa tai pehmokäynnistintä verkkoon kytketyt lämpöpumput ottavat käynnistyessään verkosta suuria hetkellisiä tehoja, mikä saattaa jännitejäykkyydeltään huonossa verkossa aiheuttaa hetkellisiä jännitteenalenemia. Lämpöpumpun ollessa taajuusmuuttajakäyttöinen, ei se aiheuta edellä mainittuja tehonkulutuksen piikkejä. Kuitenkin mikäli lämpöpumpulla korvataan jokin muu lämmitysmuoto, kuin suora sähkölämmitys, kasvattaa se kulutuspaikan sähköenergian kulutusta. Mikäli muuntopiirissä moni talous vaihtaa lämmitysmuodokseen lämpöpumpun, saattaa se aiheuttaa tarpeen sähköverkon jännitejäykkyyden parantamiselle.

Järjestelmien tuomia haasteita voidaan ratkaista ja löytää uusia järjestelmiä hyödyntäviä sovelluksia, kun kommunikaatio niiden kanssa mahdollista. Kommunikointi mahdollistetaan standardoiduilla kommunikaatiorajapinnoilla, joiden kautta järjestelmiä voidaan ohjata. Tällöin ne voidaan kytkeä esimerkiksi kiinteistöautomaatioon tai muuhun ulkoiseen ohjaukseen. Valmistajien suljetut pilvipalvelut muodostavat myös erään rajapinnan kommunikaatiolle, joihin pääsy on usein vain valmistajalla ja järjestelmän omistajalla.

Yksi suosituimmista järjestelmien ohjauksen mahdollistavista tekniikoista on Modbus. Modbus-standardi määrittelee fyysisen tason rajapinnat mahdollistaen ohjauksen joko sarja- tai Ethernet-liitäntää käyttäen, mikä mahdollistaa protokollan käyttämisen paikallisesti tai internetin yli toimivien etäyhteyksien välityksellä. Modbus-standardin määritellessä kommunikointia vain hyvin matalalla tasolla on sitä helppo soveltaa erilaisiin sovelluksiin ja sen päälle voidaan rakentaa korkeamman tason viestintä- ja ohjausprotokollia. Modbusin päälle voidaan myöskin toteuttaa yksittäisen laitteen ohjaukseen ja monitorointiin tarvittavat sovellukset tai liittäminen kotiautomaatioon osaksi älykästä kotia.

Aurinkosähköjärjestelmien ja energiavarastojen osalta työssä tarkasteltiin muutamia eri rajapintoja, joista tärkein on SunSpec Modbus. Se pohjautuu Modbus-standardiin, ja sen avulla voidaan ohjata energiavarastojen ja aurinkosähköinverttereiden toimintaa, esimerkiksi rajoittaa tehoa tai säätää loistehontuotantoa. Suurin käyttö tällä rajapinnalla on paikallisessa automaatiossa, esimerkiksi tuotantotietojen käyttö kuormanohjaukseen.

Lämpöpumppujen kohdalla korkeamman tason rajapintoja kuten SG-Ready-ohjausrajapintaa tai Saksalaista EVU-Sperre-ohjausrajapintaa käyttävät sovellukset taas keskittyvät yhden lämpöpumpun sijasta useamman pumpun muodostamaan pooliin. Lämpöpumppupoolin ohjaamista voidaan käyttää erilaisiin sovelluksiin kuten huipputehon rajoittamiseen verkossa tai sähköntuotannon ja -kulutuksen tasapainottamiseen. Aggregaattorin hallinnoimaa lämpöpumppupoolia voidaan käyttää esimerkiksi kysyntäjouston tehokkaaseen toteuttamiseen. Aggregaattorille lämpöpumppupoolin ohjaaminen taas on pääasiallista liiketoimintaa, jolla se osallistuu sähkömarkkinoille.

Työssä tarkasteltiin kuutta eri sidosryhmää, joista Fingrid on sidoksissa vain aurinkosähköjärjestelmiin. Muut sidosryhmät hyödyntävät eri kommunikaatiorajapintoja eri tarkoituksiin, joista osalle niiden käyttö voi olla myös yrityksen ydinliiketoimintaa.

Rajapinnat mahdollistavat useita sovelluksia, mutta kaikki käsitellyt sovellukset eivät vielä ole yleistyneet Suomessa. Etämonitorointi ja -hallinta on ollut jo pidempään käytössä pilvipalveluiden muodossa, ja kiinteistöautomaatio yleistyy kotitalouksissa energiansäästötarkoituksessa.

Kysyntäjousto on vasta yleistymässä Suomessa, sillä aggregaattoreita ei Suomessa oikeastaan ole. Itsenäisen aggregaattorin laajennettu pilotti käynnistyi 21.7.2020, ja siinä testataan aiemmin kokeiltujen ratkaisujen skaalautuvuutta. Fingrid on ajamassa tätä kaikille säätösähkömarkkinoille avointa pilottia eteenpäin.

Teollisuuslaitosten tarvitsemia loistehon kompensaattoreita voidaan tulevaisuudessa korvata aurinkosähköjärjestelmillä, sillä suuritehoiset invertterit voivat tuottaa nimellistehollaan myös loistehoa ympäri vuorokauden. Tällöin kalliiden loistehon kompensaattoreiden takaisinmaksuaikaa voidaan pienentää, kun sitä voidaan hyödyntää myös sähköntuotantoon.

Kommunikaatiorajapintojen hyödyntäminen vaatii vahvaa standardointia, jotta järjestelmät olisivat valmistajasta riippumattomia. Järjestelmien fyysisten rajapintojen osalta tämä alkaa olla jo hyvällä mallilla, mutta niiden hyödyntäminen vaatii vielä kehittämistä. Vielä on kysymysmerkkinä, miten esimerkiksi Suomessa aggregaattori voi ohjata poolina akullisia aurinkosähköjärjestelmiä, tai mitä rajapintaa se käyttäisi virtuaalivoimalaitoksen kautta yksittäisten lämpöpumppujen ohjaamiseen.


\printbibliography[heading=bibintoc]

%%%%% Appendices.

%\begin{appendices}

%\chapter{Esimerkkiliite}
%\label{ch:liite}
%Tämä on liite


%\end{appendices}

\end{document}
