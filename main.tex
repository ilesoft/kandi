%%%%%%%%%%%%%%%%%%%%%%%%%%%%%%%%%%%%%%%%%%%%%%%%%%%%%%%%%%%
%% Congratulations, you've made an excellent choice
%% of writing your Tampere University thesis using
%% the LaTeX system. This document attempts to be
%% as complete a template as possible to let you focus
%% on the most important part: the writing itself.
%% Thus the details regarding the visual appearance
%% and even structure have already been worked out
%% for you!
%%
%% I sincerely hope you will find this template useful
%% in completing your thesis project. I've tried to
%% add comments (followed by the % sign) to clarify
%% the structure and purpose of some of the commands.
%% Most of the magic happens in the file tauthesis.cls,
%% which you are more than welcome to take a look at.
%% Just refrain from editing it in the most crucial
%% versions of the thesis!
%%
%% I wish you and your thesis project the best of luck!
%% If this template causes you trouble along the way
%% or if you've any suggestions for improving it,
%% please contact me via email at
%%
%% ville.koljonen (at) tuni.fi.
%%
%% Yours,
%%
%% Ville Koljonen
%% 16th May 2019
%%
%% PS. This template or its associated class file don't
%% come with a warranty. The content is provided as is,
%% without even the implied promise of fitness to the
%% mentioned purpose. You, as the author of the thesis,
%% are responsible for the entire work, including the
%% provided material. No one else is liable to you for
%% any damage inflicted on you or your thesis, were it
%% caused by using this template or not.
%%%%%%%%%%%%%%%%%%%%%%%%%%%%%%%%%%%%%%%%%%%%%%%%%%%%%%%%%%%

%%%%% NOTICE %%%%%
%% Please read through the entire template
%% (files under ./tex) to find all instructions.
%% It is possible that the attached pdf files
%% do not include the latest information.
%%%%%%%%%%%%%%%%%%

%%%%% INSTRUCTIONS FOR COMPILING THE DOCUMENT %%%%%
%% Overleaf: just click Recompile.
%% Terminal:
%%  1. pdflatex main.tex
%%  2. makeindex -s main.ist -t main.glg -o main.gls main.glo
%%  3. biber main
%%  4. pdflatex main.tex
%%  5. pdflatex main.tex
%% Similar sequence of commands is also required
%% in LaTeX specific editors.
%%%%%%%%%%%%%%%%%%%%%%%%%%%%%%%%%%%%%%%%%%%%%%%%%%%

%%%%% PREAMBLE %%%%%

%%%%% Document class declaration.
% The possible optional arguments are
%   finnish - thesis in Finnish (default)
%   english - thesis in English
%   numeric - citations in numeric style (default)
%   authoryear - citations in author-year style
% Example: \documentclass[english, authoryear]{tauthesis}
%          thesis in English with author-year citations
\nonstopmode
\documentclass{tauthesis}

% The glossaries package throws a warning:
% No language module detected for 'finnish'.
% You can safely ignore this. All other
% warnings should be taken care of!

%%%%% Your packages.
% Before adding packages, see if they can be found
% in tauthesis.cls already. If you're not sure that
% you need a certain package, don't include it in
% the document! This can dramatically reduce
% compilation time.

% Graphs
\usepackage{pgfplots}
\pgfplotsset{}%compat=1.15}

% Subfigures and wrapping text
\usepackage{subcaption}

% Mathematics packages
\usepackage{amsmath, amssymb, amsthm}
%\usepackage{bm}

%%%%% Your commands.

% Print verbatim LaTeX commands
\newcommand{\verbcommand}[1]{\texttt{\textbackslash #1}}

% Basic theorems in Finnish and in English.
% Remove [chapter] if you wish a simply
% running enumeration.
\newtheorem{lause}{Lause}[chapter]
\newtheorem{theorem}[lause]{Theorem}

% Use the commented version for individually
% enumerated lemmas
\newtheorem{apulause}[lause]{Apulause}
\newtheorem{lemma}[lause]{Lemma}
% \newtheorem{apulause}{Apulause}[chapter]
% \newtheorem{lemma}{Lemma}[chapter]

% Definition style
\theoremstyle{definition}
\newtheorem{maaritelma}{Määritelmä}[chapter]
\newtheorem{definition}[maaritelma]{Definition}
% examples in this style

%%%%% Glossary information.

\loadglsentries[main]{tex/sanasto.tex}
\makeglossaries

%%%%% Citation information.

\addbibresource{tex/references.bib}

\hypersetup{hidelinks}

\begin{document}

%%%%% FRONT MATTER %%%%%

\frontmatter

%%%%% Thesis information and title page.

% The titles of the work. If there is no subtitle,
% leave the arguments empty. Pass the title in
% the primary language as the first argument
% and its translation to the secondary language
% as the second.
\title{Kuvaava otsikko}{A Descriptive Title}
\subtitle{Tarkentava alaotsikko}{A Specifying Subtitle}

% The author name.
\author{Etunimi Sukunimi}

% The examiner information.
% If your work has multiple examiners, replace with
% \examiner[<label>]{<name> \\ <name>}
% where <label> is an appropriate (plural) label,
% e.g. Examiners or Tarkastajat, and <name>s are
% replaced by the examiner names, each on their
% separate line.
\examiner{Prof. Proffa Proffanen}

% The finishing date of the thesis (YYYY-MM-DD).
\finishdate{2020}{01}{29}

% The type of the thesis (e.g. Kandidaatintyö
% or Master of Science Thesis) in the primary
% and the secondary languages of the thesis.
\thesistype{Opinnäytetyön taso}{Thesis type}

% The faculty and degree programme names in
% the primary and the secondary languages of
% the thesis.
\facultyname{Tiedekunnan nimi}{Faculty Name}
\programmename{Tutkinto-ohjelma}{Degree Programme}

% The keywords to the thesis in the primary and
% the secondary languages of the thesis
\keywords%
    {avainsana, avainsana, avainsana, avainsana, avainsana}
    {keyword, keyword, keyword, keyword, keyword}

\maketitle

%%%%% Abstracts and preface.

% Write the abstract(s) and the preface
% into a separate file for the sake of clarity.
% Pass the appropriate file name as the first
% argument to these commands. Put the \abstract
% in the primary language first and the
% \otherabstract in the secondary language second.
% Those who do not speak Finnish only need the
% first abstract. The second argument of
% the \preface command takes the place where
% the thesis was signed in.
\abstract{tex/tiivistelma.tex}
\preface{tex/alkusanat.tex}{Tampereella}

%%%%% Table of contents.

\tableofcontents

%%%%% Lists of figures, tables, listings and terms.

% Print the lists of figures and/or tables.
% (Un)comment either of these commands as required.
% Both are optional, but if there are many important
% figures/tables, listing them may be a good idea.

%\listoffigures
%\listoftables
%\lstlistoflistings

% Print the glossary of terms.

\glossary

%%%%% MAIN MATTER %%%%%

\mainmatter

% Write each of the chapters of the thesis
% into a separate file for the sake of clarity.
% They can be \input as shown below. Give both
% the chapters and their files as descriptive
% names as possible.
\chapter{Johdanto}
\label{ch:johdanto}
Lämpöpumppujen käyttö kiinteistöjen lämmitykseen kasvattaa suosiotaan. Vuoden 2000 jälkeen on Suomeen asennettu noin 700 000 lämpöpumppua\footnote{Alle \SI{26}{\kilo\watt}}\parencite{kummu}. Lämpöpumppujen kuluttama teho saattaa pienissä muuntopiireissä kasvaa hyvinkin merkittäväksi sen korvatessa vanhoja lämmitysmenetelmiä. Tämä aiheuttaa ongelmia sähkön siirrossa. Lämpöpumppuja, kuten monia muitakin sähköä käyttäviä laitteita, voidaan kuitenkin ohjata. On mahdollista toteuttaa paikallista kysyntäjoustoa ja ohjata lämpöpumppuja pientuotannon tuotantomäärien mukaan.

Lämpöpumppujen kasvanut suosio avaa myös uusia mahdollisuuksia. Suuria määriä lämpöpumppuja voidaan ohjata yhtenä ryhmänä. Ryhmää ohjaamalla voidaan toteuttaa kysyntäjoustoa jopa kansallisella tasolla ja vakauttaa säkövekon taajuutta ja jännitettä. Lämöpumppuryhmiä voidaan käyttää myös virtuaalivoimaloina ja niiden tehokapasiteettia voidaan myydä siitä maksaville\parencite{ShenJiangLi, fischerTriebelSelinger}.

Lämpöpumppuja on mahdollista siis käyttää osana älykästä sähköverkkoa. Lämpöpummpujen ohjaus voidaan mahdollistaa erilaisten rajapintojen kautta. Laitevalmistajien on tarjottava nämä rajapinnat. Rajapintoja voidaan toteuttaa monilla tavoilla ja valmiita teknisiä ratkaisuja on olemassa.



\chapter{Viittaustekniikat}
\label{ch:viittaustekniikat}
Viittaus sisältää kaksi pääkohtaa: tekstissä esiintyvän lähdeviitteen ja lähdeluettelon, jossa on jokaisen lähteen yksilöivät (bibliografiset) tiedot. Tässä osiossa esitellään 2 yleistä viittausten merkintätapaa:
\begin{enumerate}
    \item numeroviittausjärjestelmä (Vancouver-järjestelmä), esim. [1], [2], \ldots
    \item nimi-vuosijärjestelmä (Harvard-järjestelmä), esim. (Weber 2001), (Kaunisto 2003), \ldots
\end{enumerate}
Numeroviittaus sijoitetaan hakasulkeisiin ja nimi-vuosiviittaus kaarisulkeisiin. Ensin mainitussa käytetään juoksevaa numerointia ja jälkimmäisessä tekijän sukunimeä ja julkaisuvuotta. Kumpikin viittaustapa on sallittu, ja niiden yleisyys vaihtelee aloittain. Valitse yksi ja ole järjestelmällinen sitä käyttäessäsi.

\LaTeX{}in tavallisimmin käytetty lähdeviittaustoiminto on pitkään ollut Bib\TeX. Se on kuitenkin jo vanha, ja sitä joustavampi ja ilmaisuvoimaisempi vaihtoehto on Bib\LaTeX{} \parencite{biblatex}. Käytännössä suuri osa tieteellisestä julkaisemisesta hyödyntää vanhempaa työkalua, mutta muutostakin on tapahtumassa. Näistä syistä tämä pohja ohjaa käyttämään Bib\LaTeX{}ia.

Molemmat esitellyt järjestelmät perustuvat siihen, että käytettyjen lähteiden bibliografiset tiedot kerätään \texttt{.bib}-tiedostoon erityisellä syntaksilla. Ohjelma lukee sekä tämän ''tietokannan'' että kirjoitettavan dokumentin, sekä muodostaa viitteet ja viiteluettelon niiden pohjalta. Seuraavassa käydään läpi molempien viittaustyylien muodostaminen Bib\LaTeX{}in avulla. Oletuksena pohjassa on aktiivisena numeroviittaus, ja sen voi vaihtaa nimi-vuosi\-järjestelmään kirjoittamalla dokumenttiluokan valinnaiseksi argumentiksi \texttt{authoryear}.

\section{Lähdeviittaukset tekstissä}

Lähdeviittaus sijoitetaan tekstin joukkoon mahdollisimman lähelle viittauskohtaa. Pääsääntönä tekstiviittaus sijoitetaan virkkeen sisälle ennen pistettä.

\begin{quotation}
\noindent Weber väittää, että\ldots [1].

\noindent Cattaneo et al. esittävät tutkimuksessaan [2] uuden\ldots

\noindent Tuloksena on\ldots [1, s. 23]. Pitää myös huomata\ldots [1, ss. 33--36]

\noindent Esitetyn teorian mukaan\ldots (Weber 2001).

\noindent Erityisesti on huomioitava\ldots (Cattaneo et al. 2004).

\noindent Weber (2001, s. 230) on todennut\ldots

\noindent Alan kirjallisuudessa [1, 3, 5] esitetyn mukaan\ldots

\noindent Alan kirjallisuudessa [1][3][5] esitetyn mukaan\ldots

\noindent Aihetta on tutkittu ja raportoitu erittäin laajasti [6--18]\ldots

\noindent\ldots kirjallisuudessa (Weber 2001; Kaunisto 2003; Cattaneo et al. 2004) on esitetty\ldots
\end{quotation}

Lähdeluettelon pohjana toimivan \texttt{.bib}-tiedoston jokaista erillistä lähdettä varten varataan yksikäsitteinen tunniste, joka aloittaa tietojen esittelyn. Tunnisteet kannattaa valita mahdollisimman kuvaaviksi, sillä kaikki viittaukset tapahtuvat niiden avulla. Numeroviittausjärjestelmässä jokainen viittaus luodaan \verbcommand{cite}-komennolla: esimerkiksi \verbcommand{cite\{notsoshort\}}. Tämä tuottaa paikalleen vaikkapa merkinnän \cite{notsoshort}, riippuen lopullisesta lähdeluettelosta. Viittaukseen voidaan lisätä tietoja valinnaisten argumenttien avulla: esimerkiksi kirjoittamalla \verbcommand{cite[s. 30]\{notsoshort\}} tuottaa \cite[s. 30]{notsoshort} ja \verbcommand{cite[katso][s. 30]\{notsoshort\}} tuottaa \cite[katso][s. 30]{notsoshort}.

Nimi-vuosijärjestelmä on monimutkaisempi, sillä se sallii monenlaisia siteerausmahdollisuuksia, kuten yllä nähdään. Bib\LaTeX{}in toimintalogiikka pysyy kuitenkin samanlaisena, vain komennot vaihtuvat. Tärkeimmät viittauskomennot ovat \verbcommand{parencite}, \verbcommand{parencite*}, \verbcommand{citeauthor} ja \verbcommand{textcite}, jotka tuottavat tuloksinaan (Oetiker et al. 2018), (2018), Oetiker et al. ja Oetiker et al. (2018) samassa järjestyksessä.  Lisää komentoja voi etsiä dokumentaatiosta \parencite{biblatex}.

\section{Lähdeluettelo}

Lähteestä kerrotaan vähintään
\begin{itemize}
    \item tekijä(t),
    \item otsikko,
    \item julkaisuaika,
    \item julkaisija,
    \item sivunumerot (kirjat ja lehdet), sekä
    \item verkko-osoite,
\end{itemize}
jos ne tiedetään. Bib\LaTeX{} huolehtii tietojen järjestämisestä keskenään samalla tavalla. Järjestelmän käytössä on oleellista tietää myös lähteen tyyppi: lehtiartikkeli, kirja, konferenssijulkaisu, raportti ja patentti ovat vain esimerkkejä erilaisista mahdollisuuksista. Tämä tieto sisällytetään \texttt{.bib}-tiedostoon, ja muotoilu tapahtuu automaattisesti lähteen tyypin perusteella. Alla on esitetty malliksi lehtiartikkelin tietojen kirjoittaminen lähteeksi \texttt{.bib}-tiedostoon.

\texttt{
\begin{quotation}
    \noindent @article\{braams1991babel,\\
    title=\{Babel, a multilingual style-option system \\
    for use with \textbackslash LaTeX’s standard document styles\},\\
    author=\{Braams, Johannes L\},\\
    journal=\{TUGboat\},\\
    volume=\{12\},\\
    number=\{2\},\\
    pages=\{291--301\},\\
    year=\{1991\}\\
    \}
\end{quotation}
}

Eri viittausjärjestelmissä ylläoleva näkyisi lähdeluettelossa muodoissa
\begin{enumerate}[label={[\arabic*]}]
    \item J. L. Braams. Babel, a multilingual style-option system for use with \LaTeX's standard document styles. \emph{TUGboat} 12.2 (1991), 291--301.
    \item[] Braams, J. L. (1991). Babel, a multilingual style-option system for use with \LaTeX's standard document styles. \emph{TUGboat} 12.2, 291--301.
\end{enumerate}
Opinnäytteissä lähdeluettelo kannattaa järjestää aakkosjärjestykseen ensimmäisen kirjoittajan sukunimen perusteella. Tämä tapahtuu tässä pohjassa automaattisesti. Erinomainen keino muodostaa yksittäinen lähde nopeasti on etsiä sille pohja Google Scholarin avulla. Se luo automaattisesti hyvän yritteen Bib\TeX{}in ja Bib\LaTeX{}in käyttöön. Dokumentaation lisäksi hyvä yhteenveto mahdollisista lähdetyypeistä ja niihin liittyvistä kentistä löytyy lähteestä \parencite{bibmanagement}.

% Add chapters similarly.

\chapter{Yhteenveto}
\label{ch:yhteenveto}
Lämpöpumppujen ja aurinkosähköjärjestelmien merkitys sähköverkossa kasvaa niiden suosion kasvaessa. Lämpöpumput korvaavat muita lämmitys- ja viilennysmuotoja ja aurinkosähköjärjestelmillä haetaan kotitalouksissa omavaraisuutta ja ylipäätään säästöjä sähkölaskussa.

Aurinkosähköjärjestelmien määrän kasvu lisää sähköverkon tuotannon ailahtelevuutta ja säätövoiman tarvetta, sillä tuotanto vaihtelee tunneittain. Tuotantoa voidaan ennustaa sääennusteen pohjalta, mutta yksittäiset pilvet voivat aiheuttaa alueellisesti nopeitakin tehovaihteluita. Lisäksi huipputeho sijoittuu päiväsaikaan, joten ihmisten palatessa töistä ja kulutuksen kasvaessa tuotannon laskiessa samanaikaisesti säätötehon tarve kasvaa entisestään.

Lämpöpumppujen lisääntyminen ja esimerkiksi suorasta sähkölämmityksestä eroava käyttäytyminen vaikuttavat sähköverkkoon. Lämpöpumppujen tekniikka kehittyy ja nykyisin kotitalouskäyttöön myytävissä pumpuissa on yleensä suoraan verkkoon kytketyn oikosulkumoottorin tilalla taajuusmuuttajakäyttöinen oikosulkumoottori. Tämä kehitys myös osaltaan määrittelee miten verkkoon liitettyjen lämpöpumppujen määrän kasvu vaikuttaa sähköverkkoon. Ilman taajuusmuuttajaa tai pehmokäynnistintä verkkoon kytketyt lämpöpumput ottavat käynnistyessään verkosta suuria hetkellisiä tehoja, mikä saattaa jännitejäykkyydeltään huonossa verkossa aiheuttaa hetkellisiä jännitteenalenemia. Lämpöpumpun ollessa taajuusmuuttajakäyttöinen, ei se aiheuta edellä mainittuja tehonkulutuksen piikkejä. Kuitenkin mikäli lämpöpumpulla korvataan jokin muu lämmitysmuoto, kuin suora sähkölämmitys, kasvattaa se kulutuspaikan sähköenergian kulutusta. Mikäli muuntopiirissä moni talous vaihtaa lämmitysmuodokseen lämpöpumpun, saattaa se aiheuttaa tarpeen sähköverkon jännitejäykkyyden parantamiselle.

Järjestelmien tuomia haasteita voidaan ratkaista ja löytää uusia järjestelmiä hyödyntäviä sovelluksia, kun kommunikaatio niiden kanssa mahdollista. Kommunikointi mahdollistetaan standardoiduilla kommunikaatiorajapinnoilla, joiden kautta järjestelmiä voidaan ohjata. Tällöin ne voidaan kytkeä esimerkiksi kiinteistöautomaatioon tai muuhun ulkoiseen ohjaukseen. Valmistajien suljetut pilvipalvelut muodostavat myös erään rajapinnan kommunikaatiolle, joihin pääsy on usein vain valmistajalla ja järjestelmän omistajalla.

Yksi suosituimmista järjestelmien ohjauksen mahdollistavista tekniikoista on Modbus. Modbus-standardi määrittelee fyysisen tason rajapinnat mahdollistaen ohjauksen joko sarja- tai Ethernet-liitäntää käyttäen, mikä mahdollistaa protokollan käyttämisen paikallisesti tai internetin yli toimivien etäyhteyksien välityksellä. Modbus-standardin määritellessä kommunikointia vain hyvin matalalla tasolla on sitä helppo soveltaa erilaisiin sovelluksiin ja sen päälle voidaan rakentaa korkeamman tason viestintä- ja ohjausprotokollia. Modbusin päälle voidaan myöskin toteuttaa yksittäisen laitteen ohjaukseen ja monitorointiin tarvittavat sovellukset tai liittäminen kotiautomaatioon osaksi älykästä kotia.

Aurinkosähköjärjestelmien ja energiavarastojen osalta työssä tarkasteltiin muutamia eri rajapintoja, joista tärkein on SunSpec Modbus. Se pohjautuu Modbus-standardiin, ja sen avulla voidaan ohjata energiavarastojen ja aurinkosähköinverttereiden toimintaa, esimerkiksi rajoittaa tehoa tai säätää loistehontuotantoa. Suurin käyttö tällä rajapinnalla on paikallisessa automaatiossa, esimerkiksi tuotantotietojen käyttö kuormanohjaukseen.

Lämpöpumppujen kohdalla korkeamman tason rajapintoja kuten SG-Ready-ohjausrajapintaa tai Saksalaista EVU-Sperre-ohjausrajapintaa käyttävät sovellukset taas keskittyvät yhden lämpöpumpun sijasta useamman pumpun muodostamaan pooliin. Lämpöpumppupoolin ohjaamista voidaan käyttää erilaisiin sovelluksiin kuten huipputehon rajoittamiseen verkossa tai sähköntuotannon ja -kulutuksen tasapainottamiseen. Aggregaattorin hallinnoimaa lämpöpumppupoolia voidaan käyttää esimerkiksi kysyntäjouston tehokkaaseen toteuttamiseen. Aggregaattorille lämpöpumppupoolin ohjaaminen taas on pääasiallista liiketoimintaa, jolla se osallistuu sähkömarkkinoille.

Työssä tarkasteltiin kuutta eri sidosryhmää, joista Fingrid on sidoksissa vain aurinkosähköjärjestelmiin. Muut sidosryhmät hyödyntävät eri kommunikaatiorajapintoja eri tarkoituksiin, joista osalle niiden käyttö voi olla myös yrityksen ydinliiketoimintaa.

Rajapinnat mahdollistavat useita sovelluksia, mutta kaikki käsitellyt sovellukset eivät vielä ole yleistyneet Suomessa. Etämonitorointi ja -hallinta on ollut jo pidempään käytössä pilvipalveluiden muodossa, ja kiinteistöautomaatio yleistyy kotitalouksissa energiansäästötarkoituksessa.

Kysyntäjousto on vasta yleistymässä Suomessa, sillä aggregaattoreita ei Suomessa oikeastaan ole. Itsenäisen aggregaattorin laajennettu pilotti käynnistyi 21.7.2020, ja siinä testataan aiemmin kokeiltujen ratkaisujen skaalautuvuutta. Fingrid on ajamassa tätä kaikille säätösähkömarkkinoille avointa pilottia eteenpäin.

Teollisuuslaitosten tarvitsemia loistehon kompensaattoreita voidaan tulevaisuudessa korvata aurinkosähköjärjestelmillä, sillä suuritehoiset invertterit voivat tuottaa nimellistehollaan myös loistehoa ympäri vuorokauden. Tällöin kalliiden loistehon kompensaattoreiden takaisinmaksuaikaa voidaan pienentää, kun sitä voidaan hyödyntää myös sähköntuotantoon.

Kommunikaatiorajapintojen hyödyntäminen vaatii vahvaa standardointia, jotta järjestelmät olisivat valmistajasta riippumattomia. Järjestelmien fyysisten rajapintojen osalta tämä alkaa olla jo hyvällä mallilla, mutta niiden hyödyntäminen vaatii vielä kehittämistä. Vielä on kysymysmerkkinä, miten esimerkiksi Suomessa aggregaattori voi ohjata poolina akullisia aurinkosähköjärjestelmiä, tai mitä rajapintaa se käyttäisi virtuaalivoimalaitoksen kautta yksittäisten lämpöpumppujen ohjaamiseen.


%%%%% Bibliography/references.

% Print the bibliography according to the
% information in ./tex/references.bib and
% the in-line citations used in the body of
% the thesis.
% \emergencystretch=2em
\printbibliography[heading=bibintoc]

%%%%% Appendices.

% Use only if it clarifies the structure of
% the document. Remember to introduce each
% appendix and its content.

\begin{appendices}

\chapter{Esimerkkiliite}
\label{ch:liite}
Tämä on liite


\end{appendices}

\end{document}
