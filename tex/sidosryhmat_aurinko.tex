Luonnollisesti eri sidosryhmät ovat kiinnostuneet erilaisiin käyttötarkoituksiin soveltuvista rajapinnoista. Osa sidosryhmistä on kiinnostunut vain datan lukemiseen tarkoitetusta rajapinnasta käyttäen sitä esimerkiksi käyttövarmuuden ja tuoton tilastointiin, kun taas osa on kiinnostunut järjestelmän etähallinnasta esimerkiksi verkon joustavuuden paranta-miseksi.

\section{Kantaverkkoyhtiö Fingrid}
  Suomen kantaverkkoyhtiönä Fingrid asettaa vaatimuksia siihen liittyen, mitä rajapintoja järjestelmien on sisällettävä. Se vaatii alle 1MW järjestelmien (voimalaitosten) sisältämään logiikkaportin, jonka kautta saapuvalla käskyllä voimalaitoksen on lopetettava pätötehon tuotanto viiden sekunnin kuluessa. Yli 1MW:n suuruiset järjestelmät tulee taas varustaa väyläliitännällä, jolla voidaan alentaa tuotannon pätötehoa annetun ohjearvon mukaisesti. Tämän liitännän on oltava yhteensopiva IEC60870-6, IEC60870-5 tai IEC61850 -protokollan kanssa. \parencite{VJV2018}
  Yli 10MW:n järjestelmien kohdalla Fingridin Kantaverkkokeskus voi tarvittaessa pyytää järjestelmän käytöstä vastaavaa toimijaa muuttamaan pätö- tai loistehosäädön asettelu-arvoja voimalaitosteknologian asettamissa rajoissa \parencite{VJV2018}. Fingrid ei täten ota kantaa siihen, miten tämä säätäminen tapahtuu. Fingridin tarpeet järjestelmien rajapinnoista rajoit-tuvat siten erityistilanteissa järjestelmän alasajoon tai sen tehonsäätöön käytöstä vas-taavan toimijan kautta.

\section{Verkko-operaattori}
  \begin{itemize}
    \item kysyntäjouston toteuttaminen
    \item häiriötilanteissa järjestelmän hallinta
  \end{itemize}
\section{Järjestelmän omistaja}
  \begin{itemize}
    \item monitorointi
    \item muiden järjestelmien integraatio
  \end{itemize}
\section{Järjestelmätoimittaja}
  \begin{itemize}
    \item Luotettavuuden tarkastelu
    \item monitorointi
    \item uusien ratkaisujen kehittäminen
    \item huoltopalveluiden tarjoaminen monitorointia hyödyntämällä
  \end{itemize}