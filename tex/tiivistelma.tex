Aurinkosähkö- ja lämpöpumppujärjestelmien määrän kasvaessa niiden merkitys Suomen sähköverkossa kasvaa. Niillä on useita eri rajapintoja, joiden hyödyntämisellä voidaan ehkäistä kasvaneen määrän aiheuttamia ongelmia sähköverkolle. Toisaalta rajapintojen hyödyntäminen hyödyttää järjestelmien omistajia, koska niiden avulla voidaan esimerkiksi hyödyntää tuotettua aurinkosähköä tehokkaammin. Työssä tarkastellaan millaisia rajapintoja aurinkosähköjärjestelmillä ja lämpöpumpuilla on, millaisia tarpeita eri sidosryhmillä on rajapintojen osalta sekä millaisia sovelluksia rajapinnoille on.

Työn alussa käsitellään aurinkosähköjärjestelmien ja lämpöpumppujen teknistä toteutusta työn pohjustamiseksi, mutta pääpaino on kolmessa osa-alueessa: rajapinnoissa, sidosryhmissä ja sovelluksissa. Rajapinnoista syvimmin käsitellään Modbus-protokollaa, sillä sen käyttö on yleistä aurinkoinverttereissä ja lämpöpumpuissa. Lisäksi molempien järjestelmien osalta käsitellään sovellustason rajapintoja, inverttereille SunSpec Modbus ja lämpöpumpuille SG-Ready. Sidosryhmien osalta tarkastellaan käytännön esimerkkien avulla, millaisia tarpeita sidosryhmillä voisi olla rajapinnoille, ja miten he voisivat hyödyntää näitä. Sovelluksista käydään läpi niin sähköverkkoa kuin yksittäistä kuluttajaa hyödyttäviä sovelluksia, joista osa perustuu yksittäisen laitteen tarkasteluun ja osa taas useamman laitteen muodostamaan järjestelmään.

Rajapintojen hyödyntämismahdollisuudet vaativat standardointia, jotta uusien sovellusten kehittäminen on mahdollista. Tällöin voidaan myös kehittää laajempia kokonaisuuksia ohjaavia järjestelmiä, jolloin lämpöpumpuista ja aurinkoinverttereistä saadut hyödyt kasvavat. Tällä tavoin voidaan esimerkiksi pienentää huipputehoa muuntopiirin sisällä ja hyödyntää sähköverkon nykyistä kapasiteettia paremmin. Vaikka teknisesti kaikki sovellukset ovat mahdollisia toteuttaa, nykyisillä järjestelmäkomponenteilla eri toimijoiden välinen viestintä on haastavaa. Tämä johtuu siitä, että yhteistä rajapintaa eri toimijoille ei ole.
