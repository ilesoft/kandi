Tässä kandidaatintyössä käsitellään aurinkosähkö- ja lämpöpumppujärjestelmiä, sitä, kuinka niiden merkitys sähköverkossa kasvaa niiden suosion noustessa ja sitä, mitä niiden tarjoamat kommunikaatiorajapinnat mahdollistavat. Läpi käydään myös muutamia erilaisia standardien määrittelemiä kommunikaatiorajapintoja.

Kun sähkön tuotantoa kasvavissa määrin siirtyy aurinkosähköjärjestelmille, lisää se sähkön tuotannon vaihteluita ja vaikeuttaa tuotantomäärien ennustamista. Tämä osaltaan luo tarvetta säätövoimalle ja kulutusjoustolle sähköenergiajärjestelmässä.

Myös lämmitykseen ja jäähdytykseen käytettävien lämpöpumppujärjestelmien lisääntymisellä on vaikutusta sähköverkkoon. Lämpöpumppun korvatessa muita perinteisiä lämmitysmuotoja kuin suoraa sähkölämmmitystä, se yleensä kasvattaa käyttöpaikkansa energiankulutusta. Korvatessaa suoran sähkölämmityksen, lämpöpumppujärjestelmä todennäköisesti vähentää käyttöpaikan kokonaisenergiankulutusta, mutta vastavuoroisesti kasvattaa kulutettua huipputehoa.

Yksi keino vastata edellä mainittuihin haasteisiin on järjestelmien älykäs ohjaus. Järjestelmien ohjaus vaatii kommunikointia ja kommunikointi tapahtuu standardoitujen kommunikaatiorajapintojen lävitse erilaisilla protokollilla. Järjestelmien ohjaamisella voidaan mahdollistaa myös muita sovelluksia, liittyen esimerkiksi kotiautomaatioon.

Yksi suosituimmista matalan tason kommunikointiprotokollista on Modbus-standardin määrittelemä Modbus-protokolla, jonka avulla laitteet voivat kommunikoida paikallisesti tai Internetin välityksellä. Modbusin päälle voidaan rakentaa korkeamman tason viestintä- ja ohjausrajapintoja, jotka auttavat erilaisten sovellusten toteuttamista.

Aurinkosähköjärjestelmien ohjaamiseen käytetyistä rajapinnoista tärkein on Sunspec Modbus, joka rakentuu Modbus-rajapinnan päälle, ja jolla aurinkosähköjärjestelmiin liittyviä inverttereitä kyetään ohjaamaan. Lämpöpumppujärjestelmien ohjaamiseen liittyvistä rajapinnoista tässä työssä käsiteltiin tarkemmin Saksalaista SG-Ready-rajapintaa, joka määrittelee lämpöpumppujärjestelmille neljä erilaista loogisin portein ohjattua käyntitilaa.

Järjestelmien ohjauksen mahdollistamia sovelluksia ovat esimerkiksi kysyntäjousto, huipputehon rajoittaminen ja loistehon tuotanto. Edellä mainittujen sovellusten mahdollistamiseksi voi olla tarpeen ohjata järjestelmiä osana suurempaa ryhmää, eli poolia. Järjestelmäpooleja ohjaava ulkoinen taho eli aggregaattori voi osallistua poolin tuotanto- tai joustokyvyllä sähkömarkkinoille ja osaltaan huolehtia sähköenergiajärjestelmän tehotasapainosta.
