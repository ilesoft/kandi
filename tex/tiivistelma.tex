Aurinkosähköjärjestelmien ja lämpöpumppujen merkitys Suomen sähköverkossa kasvaa niiden määrän lisääntyessä. Niillä on useita eri rajapintoja, joiden hyödyntämisellä voidaan ehkäistä lisääntyneen määrän tuomia ongelmia sähköverkon kannalta. Toisaalta rajapintojen hyödyntäminen tuo myös järjestelmien omistajille hyötyä, kun niiden avulla voidaan tuoda säästöjä energiankulutuksen tai aurinkosähkön hyödyntämisen osalta.

Työn tarkoituksena on esitellä yleisimpiä kommunikaatiorajapintoja ja niitä hyödyntäviä sovelluksia, sekä tarkastella millaisia tarpeita eri sidosryhmillä on rajapintojen osalta. Työn painopiste on sidosryhmien ja sovellusten kohdalla, sillä niissä on hyödyntämismahdollisuuksien kannalta asioita, jotka ovat vasta tulossa laajemmin käyttöön Suomessa. 

Lämpöpumppujen osalta SG--Ready-rajapinta on kehitetty useiden lämpöpumppujen ohjausta varten virtuaalivoimalan tavoin. Sitä voidaan ohjata esimerkiksi Modbus-väylän kautta ja tätä väylää hyödyntävät myös aurinkoinvertterit, tosin omalla SunSpec Modbus -sovellustason toteutuksella. Aurinkoinverttereitä voidaan myös ohjata IEC 61850 ja IEEE 1815 sähköasemastandardien mukaisella kommunikaatiolla, mutta näitä ei oikeastaan ole vielä nykyisissä inverttereissä käytössä. Myös AMI-mittareita voidaan hyödyntää näiden laitteiden etäohjauksessa, mutta sen toiminta rajoittuu pääosin releohjaukseen.

Eri sidosryhmillä on useita tarpeita, joista osa hyödyntää samoja sovelluksia eri käyttötarkoituksiin. Tästä hyvänä esimerkkinä etämonitorointi; omistaja voi tarkastella tuottohistoriaa tai järjestelmän toimintaa reaaliaikaisesti, kun taas valmistaja voi hyödyntää tätä esimerkiksi huoltopalveluiden tarjoamiseen heti vikatilan havaittua. Sidosryhmät voivat olla myös hyvin sidoksissa toistensa kanssa, sillä kiinteistön voidaan ajatella olevan välillinen sidosryhmä omistajan ja laitteiston välillä ajaen omistajan etua, esimerkiksi energiansäästön osalta.

Perinteiset sovellukset järjestelmien osalta ovat järjestelmän etämonitorointi, sillä omistaja haluaa usein seurata järjestelmän toimintaa etänä. Uusia, vasta laajempaan käyttöön tulossa olevia sovelluksia ovat taas kysyntäjouston toteuttaminen ja loistehon kompensointi. 

Rajapintojen hyödyntämismahdollisuudet vaativat standardointia, jotta erilaisiin järjestelmiin voidaan helposti integroida uusia laitteita riippumatta valmistajasta. Tällöin voidaan myös kehittää laajempia kokonaisuuksia ohjaavia järjestelmiä, jolloin lämpöpumpuista ja aurinkoinverttereiden tuomat edut tehostuvat. Tällä tavoin voidaan vähentää esimerkiksi huipputehoa muuntopiirin sisällä, ja hyödyntää sähköverkon nykyistä kapasiteettia paremmin.