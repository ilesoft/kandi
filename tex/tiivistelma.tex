Aurinkosähköjärjestelmien ja lämpöpumppujen merkitys Suomen sähköverkossa kasvaa niiden määrän lisääntyessä. Niillä on useita eri rajapintoja, joiden hyödyntämisellä voidaan ehkäistä lisääntyneen määrän tuomia ongelmia sähköverkon kannalta. Toisaalta rajapintojen hyödyntäminen tuo myös järjestelmien omistajille hyötyä, kun niiden avulla voidaan tuoda säästöjä energiankulutuksen tai aurinkosähkön hyödyntämisen osalta.

Tässä työssä tarkastellaan aurinkosähköjärjestelmien ja lämpöpumppujen eri rajapintoja, niitä hyödyntäviä sidosryhmiä, sekä mahdollisia sovelluksia joissa niitä voidaan hyödyntää. Työn tarkoituksena on esitellä yleisimpiä kommunikaatiorajapintoja ja niitä hyödyntäviä sovelluksia, sekä tarkastella millaisia tarpeita eri sidosryhmillä on rajapintojen osalta. Työn painopiste on sidosryhmien ja sovellusten kohdalla, sillä niissä on hyödyntämismahdollisuuksien kannalta asioita, jotka ovat vasta tulossa laajemmin käyttöön Suomessa, esimerkiksi virtuaalivoimalat ja aggregaattori. 