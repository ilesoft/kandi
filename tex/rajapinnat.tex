\section{Smart Grid -ready}


\section{Modbus}

  Modbus on automaatiossa käytetty palvelin--asiakas-mallin viestintäprotokolla. Sen on alunperin kehittänyt vuonna 1979 yritys nimeltään Modicon käytettäväksi valmistamiens ohjelmoitavien logiikoiden eli \Gls{plc}:den väliseen viestintään. Nykyin Modicon on osa Ranskalaista Schneider Electriciä. Koska standardi on julkaistu avoimesti kaikkien käytettäväksi ja se on verrattain yksinkertainen, on se saavuttanut johtavan aseman teollisuudessa. Yleisimmille ohjelmointikielille on saatavilla modbus-standardin toteuttavat kirjastot, joiden avulla voidaan ohjelmallisesti  ohjata laitteita. Laajan levinneisyyden ansiosta modbus soveltuu monien eri valmistajien laitteiden järjestelmien kanssa käytettäväksi. Teollisuuden lisäksi protokollaa käytetään myös talo- ja kiinteistöautomaatiossa.\parencite{sousaPortugal, modbusSpec, modbusOrg}

  Vuonna 2004 Schneider Electric siirsi modbus-standardin hallinnan voittoa tavoittelemattomalle Modbus-järjestölle\footnote{Modbus Organization, Inc}. Järjestön muodostavat yksittäiset henkilöt ja automaatiolaitteita valmistavat yritykset. Järjestön tehtäviin kuuluu ylläpitää ja kehittää modbus-standardia ja siihen liittyviä documentteja sekä toimia etujärjestönä edistäen modbusin käyttöä. Modbusiin liittyvien documenttien ja julkaisujen avulla edistetään eri valmistajien järjestelmien välistä kommunikaatiota ja helpotetaan niiden integraatioita toisiinsa.  \parencite{modbusOrg}



vpn
