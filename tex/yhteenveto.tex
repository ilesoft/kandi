Lämpöpumppujen merkitys sähkoverkossa kasvaa niiden tullessa yhä suositummiksi ja niiden korvatessa muita lämmitys- ja viilennysmuotoja. Lämpöpumppujen lisääntyminen ja esimerikiksi suorasta sähkölämmityksestä eroava käytttäytyminen vaikuttavat sähköverkkoon. Lämpöpumppujen tekniikka kehittyy ja nykyisin kotitalouskäyttöön myytävissä pumpuissa on yleensä suoraan verkkoon kytketyn oikosulkumoottorin tilalla taajuusmuuttajakäyttöinen oikosulkumoottori. Tämä kehitys myös osaltaan märittelee miten verkkoon liitettyjen lämpöpumppujen määrän kasvu vaikuttaa sähköverkkoon. Ilman taajuusmuuttajaa tai pehmokäynistintä verkkoon kytketyt lämpöpumput ottavat käynnistyessään verkosta suuria hetkellisiä tehoja, mikä saattaa jännitejäykkyydeltään huonossa verkossa aiheuttaa hetkellisiä jännitteenalenemia. Lämöpumpun ollessa taajuusmuuttajakäyttöinen, ei se aiheuta edellämainittuja tehonkulutuksen piikkejä. Kuitenkin mikäli lämpöpupulla korvataan jokin muu lämmitysmuoto, kuin suora sähkölämmitys, kasvattaa se kulutuspaikan energiankulutusta. Mikäli muuntopiirissä moni talous vaihtaa lämmitysmuodokseen lämpöpumpun, saattaa se aiheuttaa tarpeen sähöverkon jännitejäykkyyden parantamiselle.
