Lämpöpumppujen ja aurinkosähköjärjestelmien merkitys sähkoverkossa kasvaa niiden suosion kasvaessa. Lämpöpumput korvaavat muita lämmitys- ja viilennysmuotoja ja aurinkosähköjärjestelmillä haetaan kotitalouksissa omavaraisuutta ja säästöjä sähkölaskussa. 

Aurinkosähköjärjestelmien määrän kasvu lisää sähköverkon tuotannon ailahtelevuutta ja säätövoiman tarvetta, sillä tuotanto vaihtelee tunneittain. Tuotantoa voidaan ennustaa sääennusteen pohjalta, mutta yksittäiset pilvet voivat aiheuttaa aluellisesti nopeitakin tehovaihteluita. Lisäksi huipputeho sijoittuu päiväsaikaan, joten ihmisten palatessa töistä ja kulutuksen kasvaessa tuotannon laskiessa samanaikaisesti säätötehon tarve kasvaa entisestään. 

Lämpöpumppujen lisääntyminen ja esimerikiksi suorasta sähkölämmityksestä eroava käyttäytyminen vaikuttavat sähköverkkoon. Lämpöpumppujen tekniikka kehittyy ja nykyisin kotitalouskäyttöön myytävissä pumpuissa on yleensä suoraan verkkoon kytketyn oikosulkumoottorin tilalla taajuusmuuttajakäyttöinen oikosulkumoottori. Tämä kehitys myös osaltaan märittelee miten verkkoon liitettyjen lämpöpumppujen määrän kasvu vaikuttaa sähköverkkoon. Ilman taajuusmuuttajaa tai pehmokäynistintä verkkoon kytketyt lämpöpumput ottavat käynnistyessään verkosta suuria hetkellisiä tehoja, mikä saattaa jännitejäykkyydeltään huonossa verkossa aiheuttaa hetkellisiä jännitteenalenemia. Lämöpumpun ollessa taajuusmuuttajakäyttöinen, ei se aiheuta edellämainittuja tehonkulutuksen piikkejä. Kuitenkin mikäli lämpöpupulla korvataan jokin muu lämmitysmuoto, kuin suora sähkölämmitys, kasvattaa se kulutuspaikan energiankulutusta. Mikäli muuntopiirissä moni talous vaihtaa lämmitysmuodokseen lämpöpumpun, saattaa se aiheuttaa tarpeen sähköverkon jännitejäykkyyden parantamiselle.

Järjestelmien tuomia haasteita voidaan ratkaista ja löytää uusia järjestelmiä hyödyntäviä sovelluksia, kun kommunikaatio niiden kanssa mahdollista. Kommunikointi mahdollistetaan standardoiduilla kommunikaatiorajapinnoilla, joiden kautta järjestelmiä voidaan ohjata. Tällöin ne voidaan kytkeä esimerkiksi kiinteistöautomaatioon tai muuhun ulkoiseen ohjaukseen. Valmistajien suljetut pilvipalvelut muodostavat myös erään rajapinnan kommunikaatiolle, joihin pääsy on usein vain valmistajalla ja järjestelmän omistajalla. 

Yksi suosituimmista järjestelmien ohjauksen mahdollistavista tekniikoista on Modbus-standardi. Standardi määrittelee fyysisen tason rajapinnat mahdolistaen ohjauksen joko sarjaliitäntää tai ethernet-liitäntää käyttäen, mikä mahdollistaa protokollan käyttämisen paikallisesti tai internetin yli toimivien etäyhteyksien välityksellä. Modbus-standardin määritellessä kommunikointia vain hyvin matalalla tasolla on sitä helppo soveltaa erilaisiin sovelluksiin ja sen päälle voidaan rakentaa korkeamman tason viestintä- ja ohjausprotokollia. Modbusin päälle voidaan myöskin toteuttaa yksittäisen laitteen ohjaukseen ja monitorointiin tarvittavat sovellukset tai liittäminen kotiautomaatioon osaksi älykästä kotia. 

Aurinkosähköjärjestelmien ja energiavarastojen osalta työssä tarkasteltiin muutamia eri rajapintoja, joista tärkein on SunSpec Modbus. Se pohjautuu Modbus-standardiin, ja sen avulla voidaan ohjata energiavarastojen ja aurinkoinverttereiden toimintaa, esimerkiksi rajoittaa tehoa tai säätää loistehontuotantoa. Suurin käyttö tällä rajapinnalla on paikallisessa automaatiossa, esimerkiksi tuotantotietojen käyttö kuormanohjaukseen.

Lämpöpumppujen kohdalla korkeamman tason rajapintoja kuten SG-Ready-ohjausrajapintaa tai Saksalaista EVU-Sperre-ohjausrajapintaa käyttävät sovellukset taas keskittyvät yhden lämpöpumpun sijasta useamman pumpun muodostamaan pooliin. Lämpöpumpupoolin ohjaamista voidaan käyttää erilaisiin sovelluksiin kuten huipputehon rajoittamiseen verkossa tai sähköntuotannon ja -kulutuksen tasapainottamiseen. Aggregaattorin hallinnoimaa lämpöpumppupoolia voidaan käyttää esimerkiksi kysyntäjouston tehokkaaseen toteuttamiseen. Aggregaattorille lämpöpumpupoolin ohjaaminen taas on pääasiallista liiketoimintaa, jolla se osallistuu sähkömarkkinoille.

Työssä tarkasteltiin kuutta eri sidosryhmää, joista Fingrid on sidoksissa vain aurinkosähköjärjestelmiin. Järjestelmän omistaja ja laitteiston valmistaja voivat hyödyntää valmistajan pilvipalvelua etäohjaukseen ja järjestelmän monitorointiin. Verkonhaltija voi taas käyttää AMI-mittareiden tarjoamaa rajapintaa huoltotöiden aikaiseen tuotannon alasajoon. Aggregaattori pystyy tarjoamaan kysyntäjoustoa sähkömarkkinoille esimerkiksi integroimalla verkkovoimalaitokseen AMI-mittareiden kuormanohjausreleitä tai eri valmistajien pilvipalveluita. Paikallisia, fyysisiä rajapintoja voi hyödyntää kiinteistöautomaatio, joka voi olla kiinteistölle räätälöitävä ohjausjärjestelmä tai vaihtoehtoisesti palveluna toimitettu valmis ohjausjärjestelmä.

Tärkeimmiksi sovelluksiksi näimme järjestelmien etäohjauksen ja -monitoroinnin, kysyntäjouston, aurinkosähkön kustannustehokkaan hyödyntämisen sekä loistehon kompensoinnin. Etäohjaus ja -monitorointi mahdollistaa tuotannon tarkkailun sekä vikatilojen selvittämisen etänä, mutta myös monien muiden sovellusten kehittämisen. Kysyntäjousto voidaan jakaa aggregaattorin hallinnoimiin verkkovoimalaitoksiin ja huipputehon rajoittamiseen. Tämä sovellus on sähköverkon kannalta tärkein, sillä se mahdollistaa uusiutuvien energianlähteiden suuremman määrän verkossa, mutta myös verkon tehokkaamman hyödyntämisen käyttöpistekohtaista tehoa säätämällä. 

Aurinkosähkön kustannustehokas hyödyntäminen on lähtöisin kuluttajan tuottohinnan ja ostohinnan erosta, jolloin tuotettu sähkö on järkevämpää hyödyntää itse kuin myydä verkkoon. Loistehon kompensaatiosta hyötyy eniten teollisuusyritykset, jotka tarvitsevat joka tapauksessa tuotantolaitteiden tehokertoimen takia loistehon kompensointia. Suurilla inverttereillä voidaan kompensoinnin lisäksi pienentää ostosähkön määrää, jolloin järjestelmän takaisinmaksuaika on lyhyempi kuin perinteisillä kompensointimenetelmillä.
