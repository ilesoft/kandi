Vuonna 1882 New Yorkissa käynnistettiin Pearl Street Stationin voimalaitos, joka käyn-nistyessään tuotti sähköä 85 asiakkaalle, ja vuoden päästä käynnistymisestä sillä oli jo 513 asiakasta (1). Tämä voimalaitos loi pohjan keskitetylle energiantuotannolle ja säh-köverkkojen kehitykselle. 

Nykyiset sähköverkot ovat pääosin suunniteltu keskitetyn energiantuotannon tarpeisiin, mutta uusiutuvien energialähteiden osuus kasvaa yhteiskunnassa jatkuvasti ja hajautu-neen energiantuotannon lisääntyessä älykkään verkon merkitys kasvaa. Vuodesta 2008 lähtien julkaisujen, joiden abstraktissa, otsikossa tai avainsanoissa mainitaan ”Smart Grid”, määrä on kasvanut huomattavasti Science Direct -julkaisualustalla.

Electric Power Research Institute (EPRI) määrittelee älykkään sähköverkon siten, että se sisällyttää tieto- ja kommunikaatioteknologiaa sähköntuotannon, -jakelun ja -kulutuksen eri osa-alueisiin. Sen tavoitteena on minimoida ympäristövaikutuksia, paran-taa luotettavuutta ja palvelua, vahvistaa markkinoita sekä vähentää kustannuksia ja pa-rantaa tehokkuutta. 

Nykyiset älykkäät aurinkosähköinvertterit toimivat osana älykästä sähköverkkoa hyö-dyntäen esimerkiksi SunSpec Alliancen avointa standardia. Tämä standardi mahdollis-taa eri valmistajien tuotteiden käytön samassa kohteessa, kun tiedonvaihto on standar-doitua ja siten toimii yhdessä muiden, samaa standardia käyttävien tuotteiden kanssa. Suurista valmistajista esimerkiksi SMA, Yaskawa ja ABB ovat mukana SunSpec Allian-cessa, ja standardi on käytössä laajalla tehoskaalalla laitteita, kuluttajapuolen muuta-mien kilowattien ratkaisuista kaupallisiin, kymmenien kilowattien inverttereihin.

Tässä työssä tarkastellaan aurinkosähköjärjestelmän komponentteja yleisellä tasolla jär-jestelmän ymmärtämiseksi ja perehdytään järjestelmän kommunikaatiorajapintoihin. Sen lisäksi tarkastellaan rajapintojen hyödyntämismahdollisuuksia eri sidosryhmien nä-kökulmasta. Sidosryhmien osalta työssä käsitellään Suomessa sijaitsevien järjestelmien kannalta relevantteja sidosryhmiä. Työssä ei ole tarkoitus käydä läpi sähköverkon ulko-puolella olevia ”off-grid” -järjestelmiä.

Luvussa 2 kerrotaan aurinkosähköjärjestelmän yleisestä rakenteesta sekä yleisesti käytetyistä komponenteista. Lisäksi luvussa tarkastellaan aurinkosähkön roolia sähkön tuottajana. Luvussa 3 tarkastellaan järjestelmien erilaisia kommunikaatio-rajapintoja, millaista dataa niistä voidaan saada mutta myös, miten niitä voitaisiin hyö-dyntää. Luvussa 4 pohditaan, mitä eri sidosryhmiä aurinkosähköjärjestelmiin liittyy ja mi-ten nämä voivat hyötyä kommunikaatiorajapinnoista. Lopuksi luvussa 5 on työn yhteenveto.