Lämpöpumppujen käyttö kiinteistöjen lämmitykseen kasvattaa suosiotaan. Vuoden 2000 jälkeen on Suomeen asennettu noin 700 000 lämpöpumppua\footnote{Alle \SI{26}{\kilo\watt}}\parencite{kummu}. Lämpöpumppujen kuluttama teho saattaa pienissä muuntopiireissä kasvaa hyvinkin merkittäväksi sen korvatessa vanhoja lämmitysmenetelmiä. Tämä aiheuttaa ongelmia sähkön siirrossa. Lämpöpumppuja, kuten monia muitakin sähköä käyttäviä laitteita, voidaan kuitenkin ohjata. On mahdollista toteuttaa paikallista kysyntäjoustoa ja ohjata lämpöpumppuja pientuotannon tuotantomäärien mukaan.

Lämpöpumppujen kasvanut suosio avaa myös uusia mahdollisuuksia. Suuria määriä lämpöpumppuja voidaan ohjata yhtenä ryhmänä. Ryhmää ohjaamalla voidaan toteuttaa kysyntäjoustoa jopa kansallisella tasolla ja vakauttaa säkövekon taajuutta ja jännitettä. Lämöpumppuryhmiä voidaan käyttää myös virtuaalivoimaloina ja niiden tehokapasiteettia voidaan myydä siitä maksaville\parencite{ShenJiangLi, fischerTriebelSelinger}.

Lämpöpumppuja on mahdollista siis käyttää osana älykästä sähköverkkoa. Lämpöpummpujen ohjaus voidaan mahdollistaa erilaisten rajapintojen kautta. Laitevalmistajien on tarjottava nämä rajapinnat. Rajapintoja voidaan toteuttaa monilla tavoilla ja valmiita teknisiä ratkaisuja on olemassa.
